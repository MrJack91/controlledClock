% !TeX encoding=utf8
% !TeX spellcheck = en-US

\chapter{Einleitung}



\section{Aufgabenstellung}

Wie funktioniert die Zeitsynchronisation zwischen verschiedenen Geräten? Was für Schwierigkeiten und Probleme gibt es dabei? - Des soll am Beispiel einer Computerbasierten Funkuhr gezeigt werden - Ziel dieser Arbeit ist es die Daten von einem DCF77-Empfänger abzugreifen und den Spezifikationen entsprechend zu interpretieren. Die empfangene Zeit wird via einem minimalen Network Time Protocol Server verschiedenen Client-Anwendungen (Web, Desktop) zur Verfügung gesellt.

\section{Umsetzungsidee}

Vom DCF77-Empfänger werden die Daten ausgelesen und gemäss den Spezifikationen interpretiert. Diese Daten werden anschliessend auf einem zu implementierenden Server bereitgestellt. Die Daten können einerseits via einem minimalen Network Time Protocol und andererseits via einem normalen Service abgerufen werden. Dadurch kann die Problemstellung bei der Zeitsynchronisation gezeigt werden. Um dies besser demonstrieren zu können, soll der Server eine Möglichkeit bieten eine künstliche Last / Verzögerungen zu simulieren.

Auf der Client-Seite werden drei verschiedene Uhren dargestellt. Eine wird via (S)NTP synchronisiert und eine normal via Web. Die dritte Uhr wird auf Knopfdruck via NTP synchronisiert und lässt sich manuell verstellen. Die Zeit kann anschliessend wieder via Knopfdruck synchronisiert werden.

Als Zusatzfeature könnte auf dem Server eine weitere Uhr implementiert werden, welche zu jeder Zeit die vom DCF77-Empfänger übermittelte Zeit anzeigt. Die Zeit auf dem Server könnte über eine weitere Uhr manipuliert werden, was Auswirkungen auf alle verbundenen Clients hätte. Auch diese kann per Knopfdruck wieder synchronisiert werden.

This \LaTeX{} template is designed for the creation of thesis documents (bachelor, master, phd) and targets both beginner and experienced users of \LaTeX{}. It supports all basic functionality and requirements of a technical document such as the inclusion of graphics, math, tables, references, bibliography and much more. In contrast to a standard LaTeX document this template not only loads all state of the art packages (\path{preamble/packages.tex}) to provide the best functions for each task, but also includes a separate document for the style/layout of the document (\path{preamble/style.tex}). It therefore tries to separate functionallity and layout as much as possible. And the best, everything is documented in the code and furthermore in a separate documentation file (\path{TemplateDocumentation.pdf})

This document shows in \cref{sec:example:tutorial} a general tutorial for \LaTeX{} with links to the documentation for further tasks. You can view the underlying code in file \path{content/demo/latextutorial.tex} or in this document in \cref{sec:example:code}.

The code of the template itself is documented in \path{TemplateDocumentation.pdf}.