% !TeX encoding=utf8
% !TeX spellcheck = de-CH

\chapter{Projektmanagement}



\section{Team}
Das Projektteam besteht aus folgenden Personen:
\begin{itemize}
\item Michael Hadorn\\
\textit{E-Mail: michael.hadorn@gmail.com, ZHAW-Kurzzeichen: hadormic} 
\item Daniel Brun \\
\textit{E-Mail: daniel.brun@gmx.net, ZHAW-Kurzzeichen: brundan1} 
\end{itemize}

Da unser Projektteam sehr übersichtlich ist legen wir nicht von Anfang an fixe Rollen innerhalb des Projektes fest. Einzelne Rollen werden sich im Laufe der Zeit von selbst herauskristallisieren. 

\section{Betreuung}
Unsere Projektarbeit wird durch folgende Personen betreut:
\begin{itemize}
\item Syrus Mozafar\\
Bereich: Methodik \& Planung \\
\textit{E-Mail: moza@zhaw.ch, ZHAW-Kurzzeichen: moza} 
\item George Brügger \\
Bereich: Technik \& Inhalt \\
\textit{E-Mail: irs.and.e@gmail.com} 
\end{itemize}

\section{Tools}
Für das Projektmanagement werden folgende Tools eingesetzt:
\begin{itemize}
\item Planung: \url{http://www.comindwork.com}\\
Projekt: \url{http://controlledclock.comindwork.com/}
\item Versionsverwaltung: \url{http://www.github.com}\\
Repository: \url{https://github.com/MrJack91/controlledClock}
\item Projektablage: \url{https://drive.google.com}
\end{itemize}


\section{Planung}

Die Grobplanung der Meilensteine und einzelnen Task wurde zu Beginn des Projektes gemacht. Die Detailplanung wird jeweils am Anfang des bearbeiteten Meilensteins und der Arbeitspakete gemacht. So können wir aus gemachten Erfahrungen lernen und die Planung immer dem wirklichen Projektfortschritt anpassen. So haben wir auch die Möglichkeit zu reagieren, falls wir sehen, dass gewisse Aufgaben länger brauchen als ursprünglich geplant. Dann werden wir eine entsprechende Repriorisierung der Tasks und gegebenenfalls der Meilensteine durchführen.

\subsection{Meilensteine}
Es wurden folgende Meilensteine definiert:

\begin{itemize}
\item 1. Abgabe (24.03.2014)
	\begin{itemize}
	\item Analyse abgeschlossen
	\end{itemize}
\item 2. Abgabe (14.04.2014)
	\begin{itemize}
	\item Server / API umgesetzt
	\item Auslesen und Interpretation der Daten (Teil 1)
	\end{itemize}
\item 3. Abgabe (05.05.2014)
	\begin{itemize}
	\item Auslesen und Interpretation der Daten des Empfängers abgeschlossen
	\end{itemize}
\item 4. Abgabe (31.05.2014)
	\begin{itemize}
	\item Client GUI Implementation abgeschlossen
	\item Umsetzung, Dokumentation und Projekt abgeschlossen
	\end{itemize}
\end{itemize}

\subsubsection{Analyse abgeschlossen}
Mit diesem Meilenstein wird die Grundlage für das gesamte Projekt gelegt. Es wird eine detaillierte Analyse der Aufgabenstellung und den Themenbereichen Zeitsynchronisation per Funk / im Internet, Network Time Protocol und weitere angrenzende Bereiche durchgeführt. Die Ergebnisse der Analyse dienen als Grundlage für die konkrete Umsetzung der späteren Meilensteine.

\subsubsection{Server / API umgesetzt}
Der Server, bzw. die API zum Bezug der Daten wird als eigenständiges Arbeitspaket angesehen und somit auch als Meilenstein definiert. Es werden die Schnittstellen der API definiert und umgesetzt. Darin enthalten ist eine "`normale"' API zur Abfrage der Zeit und eine weitere welche eine minimale Implementation des Network Time Protocols anbietet.

\subsubsection{Auslesen und Interpretation der Daten des Empfängers abgeschlossen}
Dieser Meilenstein ist die Grundlage des gesamten Projektes und beinhaltet das Arbeitspaket zum Auslesen und Interpretieren der Daten, die vom Funkempfänger geliefert werden. Der Empfänger musst zuerst beschafft und getestet werden, daher erfolgt die Abgabe dieses Meilensteins erst relativ spät gegen Ende des Projektes. Ein weiterer Grund ist die Komplexität der Anforderung die Daten via einem mitgeliefertem Treiber von der Schnittstelle auszulesen.

\subsubsection{Client GUI Implementation abgeschlossen}
Der Meilenstein mit dem Arbeitspaket zur Implementation des Client GUIs wird erst bei Projektabgabe abgegeben. Für uns ist die Implementation eines perfekten GUIs zweitrangig und wir möchten uns auf den Kern des Projektes fokussieren. Stellen wir im Verlauf des Projektes fest, dass wir uns bei der Aufwandschätzung verschätzt haben oder unvorhergesehene Schwierigkeiten auftauchen, werden wir Abstriche beim Client GUI machen, um die Kernfunktionalität so weit als möglich fertig zu stellen. Auch beim GUI liegt der Fokus zuallererst auf der Funktionalität und erst im Anschluss auf Zusatzfeatures und Effekten.

\subsubsection{Umsetzung, Dokumentation und Projekt abgeschlossen}
Mit diesem Meilenstein wird die Umsetzung, die Dokumentation und somit das ganze Projekt abgeschlossen und zur Bewertung freigegeben und eingereicht.

\section{Rückblick}
\subsection{1. Abgabe - 24.03.2014}
Während der ersten Iteration haben wir uns intensiv mit der Analyse unserer Aufgabenstellung und der dazugehörigen Themen beschäftigt.

Die Analyse der Themen gestaltet sich als sehr spannend. Bald sahen wir uns auch dem Problem gegenüber, dass wir nicht bei allen Themen alle Details im Rahmen dieses Projektes zusammentragen konnten. Daher mussten wir uns an gewissen Stellen mit einer weniger detaillierten Fassung begnügen. Die Zusammengetragenen Informationen haben unser Verständnis enorm verbessert und bilden eine solide Grundlage für die weiteren Meilensteine in unserem Projekt.

Bei den Themen \textit{Network Time Protocol} und \textit{Probleme \& Schwierigkeiten} haben wir uns entschieden nicht ganz so viel Zeit zu investieren, wie geplant war. Die \textit{Probleme \& Schwierigkeiten} wurden zum Teile bereits in anderen Kapiteln abgedeckt und wird auch im Fazit nochmals angeschnitten, daher haben wir das Kapitel nicht so ausführlich gestaltet. Beim \textit{Network Time Protocol} haben wir uns eine vereinfachte Variante beschränkt, da es in seiner Gesamtheit sehr komplex ist, würde dies in Kombination mit den übrigen Zielstellungen den Rahmen dieser Projektarbeit sprengen.

\subsubsection{Aktueller Planungsstand}
Für den ersten Meilenstein stand relativ wenig Zeit zur Verfügung. Nichtsdestotrotz waren wir in der Lage die geplanten Tasks in der zur Verfügung stehenden Zeit erfolgreich abzuschliessen. Auch die geschätzten Aufwände entsprachen etwa den tatsächlichen Aufwänden. Im Bereich NTP haben wir weniger Zeit gebraucht als ursprünglich geplant. Die Gründe dafür wurden oben bereits erläutert.

\subsubsection{Ausblick 2. Abgabe}
Da wir keine grossen Restanzen aus der ersten Iteration haben, gibt es auch keine direkten Konsequenzen auf die 2. Iteration. Die Analyse hat einige Knackpunkte offen gelegt, die bei der Umsetzung berücksichtigt werden müssen. Bezüglich Dokument-Layout haben wir noch kleinere offene Punkte, welche wir im Laufe der nächsten Iterationen bereinigen werden.

\subsection{2. Abgabe - 14.04.2014}
Nach Abschluss der ersten Iteration hatten wir mit Syrus Mozafar und George Brügger je ein separates Treffen bezüglich dem aktuellen Projektstand. Aus beiden Treffen konnten wir wertvolle Inputs für den weiteren Projektverlauf mitnehmen. 

Beim Treffen mit George Brügger sind wir zum Schluss gekommen, dass wir ein Replanning und weitere Verfeinerung der Aufgabenstellung machen müssen. Die Dokumentation und der Zeitplan wurden entsprechend angepasst. Auch kamen wir zum Schluss, dass eine Implementation des Backend in C unumgänglich ist. Da wir beide noch nie grössere, bzw. komplexere Programme in C geschrieben haben, stellt dies für uns eine weitere Herausforderung dar. Als Konsequenz davon rechen wir damit, dass wir nicht alle (Zusatz-)Funktionen wie geplant umsetzen können.

Aktuell stellt vor allem das Auslesen des System-Tics eine grosse Hürde für uns dar. Nach sehr intensiven und langwierigen Recherchen und einigen Dutzend ausprobierten Beispielcodes sind wir zum Schluss gekommen, dass der System-Tic nur über eine Interrupt Service Routine ausgelesen werden kann. Die Interrupt Service Routine wiederum muss als Treiber implementiert werden, da auf die entsprechenden Bereiche des Betriebsystems aus dem User-Mode heraus nicht zugegriffen werden kann. Die Ausnahme bildet hier eine allfällige Implementation in Assembler.

Da wir beide über unterschiedliche Notebooks und Betriebssysteme (Win 7 und Mac OS) verfügen, haben wir einen Versuch mit einem Raspberry Pi gestartet. Mit dem Raspberry verfügen wir über die gleichen Voraussetzungen und haben im Detail die Kontrolle, bzw. das Wissen, über die verwendete Hardware. So können wir uns bei der Implementation auch auf ein OS und eine Hardware beschränken. Niemand von uns hat bisher jemals auf diesen Ebenen programmiert. Dies ist eine spannende Herausforderung, jedoch ist Ungewiss, ob wir unser Ziel auch wirklich erreichen können.

Auch beim "`Read-"' und "`Decode-Teil"' haben wir uns ein wenig überschätzt. Zur Zeit der ersten Planung wussten wir noch gar nicht, ob wir das Backend in C oder Java umsetzen würden.
Das Read und Decode war einiges komplizierter als wir uns das vorgestellt haben. Da war zum Beispiel die fortlaufende Suche nach korrekten Zeitstempeln. Momentan haben wir zur Definition eines korrekten Zeitstempels "`nur"' die Kontrolle der Parität eingebaut, dewegen erkennen wir fast viermal mehr Zeitstempel, als es gültige Zeitstempeln im Signal hat. Dies muss verbessert werden.

\subsubsection{Aktueller Planungsstand}
Auch der Termin für die zweite Abgabe des Meilensteins ist schneller da gewesen, als wir uns das gedacht hatten. Vor allem, da wir etwas mit der Programmiersprache zu kämpfen hatten. Vom String-Handling, über Speicherallozierung bis hin zu Problemen mit Libraries, die es nur für bestimmte OS-Versionen, OS-Typen oder Compiler gibt. Die Probleme waren grundsätzlich sehr spannend und herausfordernd. Jedoch mussten wir in die Problemfindung einige Zeit investieren.


\subsection{3. Abgabe - 05.05.2014}


\subsubsection{Aktueller Planungsstand}
