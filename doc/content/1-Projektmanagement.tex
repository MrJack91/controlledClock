% !TeX encoding=utf8
% !TeX spellcheck = de-CH

\chapter{Projektmanagement}




\section{Tools}
Für das Projektmanagement werden folgende Tools eingesetzt:
\begin{itemize}
\item Planung: \url{http://www.comindwork.com}\\
Projekt: \url{http://controlledclock.comindwork.com/}
\item Versionsverwaltung: \url{http://www.github.com}\\
Repository: \url{https://github.com/MrJack91/controlledClock}
\item Projektablage: \url{https://drive.google.com}
\end{itemize}


\section{Planung}

\subsection{Meilensteine}
Es wurden folgende Meilensteine definiert:

\begin{itemize}
\item 1. Abgabe (24.03.2014)
	\begin{itemize}
	\item Analyse abgeschlossen
	\end{itemize}
\item 2. Abgabe (14.04.2014)
	\begin{itemize}
	\item Server / API umgesetzt
	\end{itemize}
\item 3. Abgabe (05.05.2014)
	\begin{itemize}
	\item Auslesen und Interpretation der Daten des Empfängers abgeschlossen
	\end{itemize}
\item 4. Abgabe (31.05.2014)
	\begin{itemize}
	\item Client GUI Implementation abgeschlossen
	\item Umsetzung, Dokumentation und Projekt abgeschlossen
	\end{itemize}
\end{itemize}

\subsubsection{Analyse abgeschlossen}
Mit diesem Meilenstein wird die Grundlage für das gesamte Projekt gelegt. Es wird eine detaillierte Analyse der Aufgabenstellung und den Themenbereichen Zeitsynchronisation per Funk / im Internet, Network Time Protocol und weitere angrenzende Bereiche durchgeführt. Die Ergebnisse der Analyse dienen als Grundlage für die konkrete Umsetzung der späteren Meilensteine.

\subsubsection{Server / API umgesetzt}
Der Server, bzw. die API zum Bezug der Daten wird als eigenständiges Arbeitspaket und somit auch als Meilenstein definiert. Es werden die Schnittstellen der API definiert und umgesetzt. Darin enthalten ist eine "`normale"' API zur Abfrage der Zeit und eine weitere welche eine minimale Implementation des Network Time Protocols anbietet.

\subsubsection{Auslesen und Interpretation der Daten des Empfängers abgeschlossen}
Dieser Meilenstein ist die Grundlage des gesamten Projektes und beinhaltet das Arbeitspaket zum Auslesen und Interpretieren der Daten, die vom Funkempfänger geliefert werden. Der Empfänger musst zuerst beschafft und getestet werden, daher erfolgt die Abgabe dieses Meilensteins erst relativ spät gegen Ende des Projektes. Ein weiterer Grund ist die Komplexität der Anforderung die Daten via einem mitgeliefertem Treiber von der Schnittstelle auszulesen.

\subsubsection{Client GUI Implementation abgeschlossen}
Der Meilenstein mit dem Arbeitspaket zur Implementation des Client GUIs wird erst bei Projektabgabe abgegeben. Für uns ist die Implementation eines perfekten GUIs zweitrangig und wir möchten uns auf den Kern des Projektes fokussieren. Stellen wir im Verlauf des Projektes fest, dass wir uns bei der Aufwandschätzung verschätzt haben oder unvorhergesehene Schwierigkeiten auftauchen, werden wir Abstriche beim Client GUI machen, um die Kernfunktionalität so weit als möglich fertig zu stellen. Auch beim GUI liegt der Fokus zuallererst auf der Funktionalität und erst im Anschluss auf Zusatzfeatures und Effekten.

\subsubsection{Umsetzung, Dokumentation und Projekt abgeschlossen}
Mit diesem Meilenstein wird die Umsetzung, die Dokumentation und somit das ganze Projekt abgeschlossen und zur Bewertung abgegeben.

\subsection{Rückblick}
\subsubsection{1. Abgabe - 24.03.2014}
%Stichworte: Bereits bei der Planung: Sportliche Timeline (2 Wochen zeit für ersten Meilenstein)
