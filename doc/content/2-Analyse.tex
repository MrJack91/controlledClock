% !TeX encoding=utf8
% !TeX spellcheck = de-CH

\chapter{Analyse}

\section{Allgemein}
Wieso braucht es Zeitsynchronisation? Was für Probleme und Schwierigkeiten können dabei auftreten? Diese Fragen aus der Aufgabenstellung werden in den nachfolgenden Kapiteln detailliert beantwortet.

\subsection{Hintergründe}
In der heutigen Zeit sind wir es uns gewohnt, dass wir uns darauf verlassen können das unsere Uhren in der Regel die gleiche Zeit anzeigen. 
Wie war das früher? Heute würde vieles nicht mehr funktionieren, wenn wir keine Uhren hätten, bzw. die Uhren nicht synchronisiert werden, oder etwa doch?

\subsubsection{Kurzer Rückblick in die Geschichte}
%http://www.astro-siggi.de/tutorial-funkuhr.html
%http://de.wikipedia.org/wiki/Geschichtliche_Entwicklung_der_Zeit%C3%BCbertragung_per_Funk

\subsubsection{Was wäre wenn...}
%Stichwort: GPS, Börse, ÖV, 

\subsection{Was für Schwierigkeiten können dabei auftreten?}

\subsection{Atomuhr}
Defition:
\begin{quote}
Eine Atomuhr ist ein Zeitmesser, dessen Zeitnormal die hochfrequente Schwingungsdauer bestimmter Atome ist (z.B. Caesium, Rubidium), die durch ein elektromagnetisches Feld oder optisches Pumpen zu Schwingungen angeregt werden und einen Quarzgenerator synchronisieren.
\end{quote} % http://www.meinberg.de/german/info/atomic_clock.htm
Die Atomuhr wurde vom Amerkikaner Isidor Isaac Rabi (1898-1988) erfunden. 1944 wurde der Erfinder mit dem Nobelpreis belohnt.
%Quelle http://www.meinberg.de/german/info/atomic_clock.htm

\subsection{Funkuhr}
Im Gegensatz zur Atomuhr, kann eine Funkuhr nicht selbst die korrekte Zeit berechnen. Sie hat einen Funkempfänger eingebaut, der von einen "`echten"' Atomuhr das Funksignal empfangen kann und darus die genaue Uhrzeit annäheren kann. Der erstmalige Zeitabgleich dauert jenach Synchronisationstyp wenige Minuten. Sobald der Decode Prozess abeschlosse ist, läuft die Uhr synchronisiert.

Die Genauigkeit einer Funkuhr liegt im zwischen 5 und 25 msec.

%Quartzuhren...
%http://de.wikipedia.org/wiki/Funkuhr

\section{Möglichkeiten der Zeitsynchronisation}
Grundsätzlich stehen mit dem heutigen Stand der Technik die nachfolgend im Detail beschriebenen Möglichkeiten zur Zeitsynchronisation zur Verfügung.

\subsection{Funk}
Funk, bwz. Funkwellen sind allgegenwärtig und aus dem heutigen Alltag nicht mehr Weg zu denken.

\subsection{Zeitzeichensender}
Jeder Zeitzeichensender verfügt über ein eindeutiges Kennzeichen...
%dUT1: Differenz Erdrotationszeit und Atomzeit	
%http://de.wikipedia.org/wiki/Zeitzeichensender

\subsection{DCF77}
DCF77 bezeichnet einen Zeitzeichensender in Deutschland (Mainflingen in Mainhausen) der von der Physikalisch-Technischen Bundesanstalt in Braunschweig (PTB) entwickelt wurde. Der Betrieb erfolgt durch die Media Broadcast GmbH. Die Bezeichnung setzt sich aus D für Deutschland, C für Langwellensender, F für Frankfurt (Mainhausen liegt in der Nähe von Frankfurt) und 77 für die Sendefrequenz (77.5 kHz).

Langwellensender

Der DCF77 sendet im Sekundenrythmus die aktuelle mitteleuropäische Zeit, bzw. mitteleuropäische Sommerzeit. Seit 1973 werden Datum und Uhrzeit gesendet.

Die ausgesendete Zeit wird mit einer am PTB entwickelten Steuereinrichtung auf Basis von drei Atomuhren ermittelt. Das daraus erhaltene Signal wird mit den primären Atomuhren des PTB synchronisiert. Aktuell sind dort zwei Caesium-Uhren und zwei Caesium-Fontänen im Einsatz. Das Endsignal hat eine Genauigkeit von $10^{-12}$. Daraus resultiert ein Fehler von ca. einer Sekunde in 30'000 Jahren.

Das DCF77 Signal hat je nach Wetterlage, Tages- und Jahreszeit eine Reichweite von 2'000 km.
%Seit 1959: Normalfrequenz (Eichfrequenz, hochgenaue Frequenz, häufig identisch mit Trägerfrequenz von Zeitzeichendiensten und Rundfunksendern im Lang- und Mittelwellenbereich)
%DCF-77
%http://www.cyber-sciences.com/support/technical_dcf77.html
%http://de.wikipedia.org/wiki/DCF77
%http://www.dcf77.de/
%http://www.bmvit.gv.at/telekommunikation/publikationen/infoblaetter/downloads/042013.pdf

\subsection{GPS}
Nebst der Zeitsynchronisation über Funk, kann eine genaure Information aus den GPS-Satelliten gewonnen werden. Dies obwhol GPS-Systeme hauptsächlich zur Lokalisierung des Standortes entwickelt wurden.
Momentan befinden sich sechs solche Satelliten auf ungefähr 200'000km Höhe. Jeder umrundet die Erde zweimal pro Tag. Auf jedem Satellit sind jeweils zwei Atomuhren vorhanden.
Ein Satellit sendet dauernd seine Bahnposition und die genaue Uhrzeit. Durch Signale mehrere Satelliten, kann der Empfänger seinen genauen Standort ermitteln.
Anschliessend kann die Laufzeit des Signals zurück gerechnet werden und die durchs Versenden verstrichene Zeit, das Delay, approximiert werden.
Durch dieses Verfahren kann die Uhrzeit, mit einer Genauigkeit unter 1${\mu}sec$, berechnet werden.

Vorteile der Zeit-Synchronisation über GPS sind: weltweilte Abdeckung und hohe sehr hohe Genauigkeit.

Ein grosser Nachteile der GPS Synchronisation ergibt sich jedoch durch das verwendete Kurzwellen Signal (zwischen 1176,45 bis 1575,42 MHz). Dieses Signal kann nur unter freien Himmel empfangen werden und eignet sich daher nicht für einen typischen Computer.

%GPS: http://www.mono.rgbtechnology.pl/de/faq/gps-zeitsynchronisation.html
%http://www.stardado.de/1244, http://www.emsec.rub.de/media/crypto/attachments/files/2010/04/ms_michael_ziaja.pdf
%GPS vs DCF77: http://www.hopf.com/de/dcf77-gps_de.html


\subsection{Internet}
Will man eine Uhr via Internet synchronisieren, stösst man rasch auf folgendes Problem:

\begin{verse}
Die genaue Senddauer eines  Datenpaktes, welches die der korreketen Uhrzeit beinhaltet, ist weder vorhersehbar noch kan man die verstrichene Zeit beim Empfänger zurückrechnen.
\end{verse}

Da keine Aussagen über die Genauigkeit der empfangenen Information möglich ist, verliert sie drastisch an Wert.

\section{Zeitprotokoll}
Um eine brauchbare Synchronisation via Internet zu ermöglichen wurde folgender, vereinfachter Datenaustausch definiert: (Der Client synchronisiert seine Zeit mit der genauen Zeit eines Server. Die Serverzeit entspricht also nicht der Clientzeit.)

\begin{enumerate}
\item Der Client schickt eine \textbf{Anfrage an den Server} und \textbf{merkt} sich seine aktuelle, lokale Zeit (welche mit grosser Wahrscheinlichkit nicht der korrekten Serverzeit entspricht).
\item Der Server antwortet mit seiner lokalen, korrekten \textbf{Zeit des Empfangs und der des Rückversandes}.
\item Der Client merkt sich die Zeit des Empfangs der Antwort. 
\end{enumerate}

Der Client hat in diesem Zustand folgende vier versch. Zeiten zur Verfügung:
\begin{enumerate}
\item Client: lokale Zeit bei der Anfrage
\item Server: Empfangszeit
\item Server: Versandzeit
\item Client: Empfangszeit
\end{enumerate}

Nun kann der Client mit folgender Rechnung die Uhrzeit gut annähern inkl. einer Fehlerabschätzung.



\section{(S)NTP}




% http://stackoverflow.com/questions/1228089/how-does-the-network-time-protocol-work
% http://www.eecis.udel.edu/~mills/database/reports/ntp4/ntp4.pdf
% http://www.emsec.rub.de/media/crypto/attachments/files/2010/04/ms_michael_ziaja.pdf