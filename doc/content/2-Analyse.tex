% !TeX encoding=utf8
% !TeX spellcheck = de-CH

\chapter{Analyse}

\section{Allgemein}
Wieso braucht es Zeitsynchronisation? Was für Probleme und Schwierigkeiten können dabei auftreten? Diese Fragen aus der Aufgabenstellung werden in den nachfolgenden Kapiteln detailliert beantwortet.

\subsection{Hintergründe}
In der heutigen Zeit sind wir es uns gewohnt, dass wir uns darauf verlassen können das unsere Uhren in der Regel die gleiche Zeit anzeigen. 
Wie war das früher? Heute würde vieles nicht mehr funktionieren, wenn wir keine Uhren hätten, bzw. die Uhren nicht synchronisiert werden, oder etwa doch?

\subsubsection{Kurzer Rückblick in die Geschichte}
%http://www.astro-siggi.de/tutorial-funkuhr.html
%http://de.wikipedia.org/wiki/Geschichtliche_Entwicklung_der_Zeit%C3%BCbertragung_per_Funk

\subsubsection{Was wäre wenn...}
%Stichwort: GPS, Börse, ÖV, 

\subsection{Was für Schwierigkeiten können dabei auftreten?}

\subsection{Atomuhr}
%http://www.meinberg.de/german/info/atomic_clock.htm

\subsection{Funkuhr}
%Quartzuhren...
%http://de.wikipedia.org/wiki/Funkuhr

\section{Möglichkeiten der Zeitsynchronisation}
Grundsätzlich stehen mit dem heutigen Stand der Technik die nachfolgend im Detail beschriebenen Möglichkeiten zur Zeitsynchronisation zur Verfügung.

\subsection{Funk}
Funk, bwz. Funkwellen sind allgegenwärtig und aus dem heutigen Alltag nicht mehr Weg zu denken.

\subsection{Zeitzeichensender}
Jeder Zeitzeichensender verfügt über ein eindeutiges Kennzeichen...
%dUT1: Differenz Erdrotationszeit und Atomzeit	
%http://de.wikipedia.org/wiki/Zeitzeichensender

\subsection{DCF77}
DCF77 bezeichnet einen Zeitzeichensender in Deutschland (Mainflingen in Mainhausen) der von der Physikalisch-Technischen Bundesanstalt in Braunschweig (PTB) entwickelt wurde. Der Betrieb erfolgt durch die Media Broadcast GmbH. Die Bezeichnung setzt sich aus D für Deutschland, C für Langwellensender, F für Frankfurt (Mainhausen liegt in der Nähe von Frankfurt) und 77 für die Sendefrequenz (77.5 kHz).

Langwellensender

Der DCF77 sendet im Sekundenrythmus die aktuelle mitteleuropäische Zeit, bzw. mitteleuropäische Sommerzeit. Seit 1973 werden Datum und Uhrzeit gesendet.

Die ausgesendete Zeit wird mit einer am PTB entwickelten Steuereinrichtung auf Basis von drei Atomuhren ermittelt. Das daraus erhaltene Signal wird mit den primären Atomuhren des PTB synchronisiert. Aktuell sind dort zwei Caesium-Uhren und zwei Caesium-Fontänen im Einsatz. Das Endsignal hat eine Genauigkeit von $10^-12$. Daraus resultiert ein Fehler von ca. einer Sekunde in 30'000 Jahren.

Das DCF77 Signal hat je nach Wetterlage, Tages- und Jahreszeit eine Reichweite von 2'000 km.
%Seit 1959: Normalfrequenz (Eichfrequenz, hochgenaue Frequenz, häufig identisch mit Trägerfrequenz von Zeitzeichendiensten und Rundfunksendern im Lang- und Mittelwellenbereich)
%DCF-77
%http://www.cyber-sciences.com/support/technical_dcf77.html
%http://de.wikipedia.org/wiki/DCF77
%http://www.dcf77.de/
%http://www.bmvit.gv.at/telekommunikation/publikationen/infoblaetter/downloads/042013.pdf

\subsection{GPS}
%GPS: http://www.mono.rgbtechnology.pl/de/faq/gps-zeitsynchronisation.html
%http://www.stardado.de/1244, http://www.emsec.rub.de/media/crypto/attachments/files/2010/04/ms_michael_ziaja.pdf
%GPS vs DCF77: http://www.hopf.com/de/dcf77-gps_de.html
\subsection{Internet}
\subsubsection{(S)NTP}