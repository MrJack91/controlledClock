% !TeX encoding=utf8
% !TeX spellcheck = de-CH

\chapter{Einleitung}



\section{Aufgabenstellung}

Wie funktioniert die Zeitsynchronisation zwischen verschiedenen Geräten? Was für Schwierigkeiten und Probleme gibt es dabei? - Das soll am Beispiel einer computerbasierten Funkuhr gezeigt werden. Ziel dieser Arbeit ist es die Daten von einem DCF77-Empfänger abzugreifen und den Spezifikationen entsprechend zu interpretieren. Die empfangene Zeit wird via einem minimalen Network Time Protocol Server verschiedenen Client-Anwendungen (Web, Desktop) zur Verfügung gestellt.

\section{Umsetzungsidee}

Vom DCF77-Empfänger werden die Daten ausgelesen und gemäss den Spezifikationen interpretiert. Diese Daten werden anschliessend auf einem zu implementierenden Server bereitgestellt. Die Daten können einerseits via einem minimalen Network Time Protocol und andererseits via einem normalen Service abgerufen werden. Dadurch kann die Problemstellung bei der Zeitsynchronisation gezeigt werden. Als optionales Feature könnte auf dem Server eine Möglichkeit geboten werden, eine künstliche Last, bzw. eine Verzögerung, zu simulieren, damit die Problemstellung besser ersichtlich wird.

Auf der Client-Seite werden drei verschiedene Uhren dargestellt. Eine wird via (S)NTP synchronisiert und eine normal, via Web. Die dritte Uhr wird auf Knopfdruck via NTP synchronisiert und lässt sich manuell verstellen. Eine erfolgreiche Synchronisation wird durch eine aufleuchtende LED und einen Timestamp angezeigt. Es findet in regelmässigen Zeitintervallen eine Synchronisation statt. Wurde auf der Server-Seite eine Zeitsynchronisation erfolgreich durchgeführt, wird dies im Client entsprechend angezeigt.

Die gesamte Implementation erfolgt in C. Dies hat den Grund, dass gewisse Möglichkeiten in den höheren Programmiersprachen wie Java nicht, bzw. nur sehr umständlich realisierbar sind. Auf der Server-Seite wird eine eigenständige Uhr implementiert, welche sich auf den System-Tic abstützt. Der System-Tic wird ca. 18.2 mal pro Sekunde erhöht, was die Implementation einer sehr präzisen Uhr ermöglicht. 
