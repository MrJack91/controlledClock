% !TeX encoding=utf8
% !TeX spellcheck = de-CH

\chapter{Schlusswort}
\label{chap:Schlusswort}

\section{Summary}
Mit dem Ende der Iteration 4 konnten wir unser Projekt über "`Funkuhren"' erfolgreich abschliessen. Während der Projektphase wurde diese schriftliche Arbeit verfasst und die Anforderungen in zwei Applikationen (Client und Server) umgesetzt. Trotz vielen Herausforderungen ist es uns gelungen eine voll funktionsfähige und einsatzbereite Applikation zu entwickeln.

\section{Abweichungen zur Aufgabenstellung und zur Umsetzungsidee}
Zu Beginn des Projektes wurde eine Aufgabenstellung und eine Umsetzungsidee definiert. Im Laufe des Projektes hat sich herauskristallisiert, dass wir nicht alle Themen in dieser Projektarbeit in der gewünschten Tiefe bearbeiten können. Zum einen mussten wir darauf verzichten als Basis für die Uhr-Implementation (Clock-Modul) den System-Tic zu verwenden und zum anderen mussten wir darauf verzichten das \textit{Network Time Protocol} in seiner kompletten Form zu implementieren. Anstelle des \textit{Network Time Protocol} haben wir eine vereinfachte Version, das \textit{Simple Network Time Protocol} Clientseitig implementiert. Ein weiterer grosser Punkt ist die Möglichkeit der Zeit-Simulation, die wir nicht implementieren konnten. Die Zeit-Simulation (Client und Server) hätte die Komplexität nochmals enorm erhöht und einiges an Zeit in Anspruch genommen.

Aus zeitlichen Gründen mussten wir auch auf der Seite des Client-GUI's einige Abstriche in Kauf nehmen. So haben wir für die Darstellung einige Standard-Plugins eingesetzt, damit wir uns auf die Implementation der Kernfunktionalitäten konzentrieren konnten. Die grafischen Effekte wie eine rotierende Uhr beim Zeit Nachstellen mussten wir auch weglassen.


\section{Fazit \& Reflexion}
Auf die vergangenen drei Monate zurückblickend sind wir sehr zufrieden mit dem Verlauf des Projektes, unseren Leistungen und dem daraus entstandenen Produkt. Es war ein äusserst spannendes, interessantes und herausforderndes Projekt. Da wir die Möglichkeit hatten die Aufgabenstellung mehrheitlich selbst zu bestimmen, waren wir in der Lage ein Aufgabenstellung und Umsetzungsidee zu finden, die uns interessierte und laufend herausforderte. Uns war nicht von Beginn an klar, dass wir während dem Projekt so viele neue spannende Sachen (z.B. Programmierung in C, Ansprechen von USB-Geräten via Treiber, Implementation eines rudimentären Web-Servers) lernen mussten, bzw. durften.

Der Aufwand für das Projekt hat sich aus unserer Sicht definitiv gelohnt. Zum einen konnten wir wertvolle Erfahrungen im Bereich der Projektplanung, Projektdurchführung und Projektdokumentation sammeln. Zum anderen konnten wir unser Wissen und unsere Fähigkeiten durch die steten Herausforderungen ständig erweitern.

\section{Danksagung}
Wir möchten uns an dieser Stelle herzlich bei folgenden Personen für Ihre Unterstützung bedanken:\\
\vspace{0.5cm}

\begin{tabular}{p{3cm} p{10cm}}
George Brügger & In der Funktion des Auftragsgebers erhielten wir nebst Anregungen und Wünschen auch stets wertvolle und hilfreiche Ideen und Tipps, wie wir gewisse Funktionalitäten umsetzen könnten. Herr Brügger begleitete uns auf der technischen Seite vom Anfang (Grundidee definieren, Ideen zu GUI und Funktionalität) bis zum Schluss.\\
Syrus Mozafar & In der Funktion des Projektmentors durften wir in regelmässigen Abständen sein Feedback zu unserem Projekt und im speziellen zur Projektplanung entgegen nehmen.\\
Sina Masquiren & Grundidee und Inspiration des Funkuhr-Projektes.\\
Jonathan Hadorn & Korrekturlesung\\
Alexandra Ganz & Korrekturlesung\\
\end{tabular}
