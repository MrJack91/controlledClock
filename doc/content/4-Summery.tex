% !TeX encoding=utf8
% !TeX spellcheck = de-CH

\chapter{Schlusswort}
\label{chap:Schlusswort}
\section{Aktueller Projektstand}
Das Projekt kann als funktionierend und abgeschlossen betrachtet werden. Selbstverständlich gäbe es noch weitere nützliche und interessante Erweiterungen, welche das Projektvolume aber deutlich sprengen würden.

\subsection{Fazit}
Wir sind sehr zufrieden mit unseren Leistungen und dem daraus folgenden Produkt. Es war ein äusserst spannendes, interessante und herausforderndes Projekt. Durch die Möglichkeit die Aufgabenstellung mehrheitlich selbst zu bestimmen, konnten wir eine Umsetzung abschliessen, die ganz in unserem Interessen lag. Dazu konnten wir Technologien verwenden, in denen wir unsere Fähigkeiten verbessern wollten.

Der Aufwand hat uns Spass gemacht und sich für uns schon nur wegen dem neuen Lerninhalt gelohnt.


% \subsection{Was für Schwierigkeiten können dabei auftreten?}
%TODO: Michi: Ich denke wie können dass lassen. Zwar interessanter Artikel, aber nicht wirklich was mit unserer Arbeit zu tun. Wir haben schon sehr viele Infos (meine dies vorallem auf den Artikel bezogen) -> http://diepresse.com/home/spectrum/zeichenderzeit/695559/Auf-der-Hohe-der-Zeit
%TODO: Dani: En Teil chömmer evtl. is Fazit neh, denke münd schono irgendwo ha wo d Schwirigkeite / Problem sind und wases für euses Projekt bedütet. - Ist dies im Fazit? - Pendent



\subsection{Reflexion}



\subsection{Danksagung}
Wir möchten uns herzlich bei folgenden Personen bedanken:
\begin{tabular}{l l}
George Brügger & Von der Rolle des Auftragsgebers erhielten wir wertvolle Inputs, Tipps, Wünsche und Anregungen wie wir was umsetzen können. Herr Brügger begleitete uns auf der technischen Seite vom Anfang (Grundidee definieren, GUI Ideen) bis am Schluss.\\
Syrus Mozafar & Als Rolle des Projektkontrolleurs durften wir in regelmässigen Abständen seine Meinung zu unserer Durchführung des Projekt in Anspruch nehmen.\\
\end{tabular}
